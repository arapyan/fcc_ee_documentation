\documentclass[preprintnumbers,nofootinbib,noshowpacs,eqsecnum,prd,superscriptaddress,letterpaper]{revtex4}

\usepackage[utf8]{inputenc}
\usepackage{graphicx}
\usepackage{amsmath,amssymb}
\usepackage{feynmp}
\usepackage{url}
\usepackage{ulem}
\usepackage{multirow}
\usepackage{hyperref,color}

\renewcommand{\sfdefault}{phv}
\renewcommand{\baselinestretch}{1.2}

\DeclareGraphicsRule{*}{mps}{*}{}

\input{declare}

\usepackage{booktabs,siunitx}
\usepackage{multirow}
\usepackage{tabularx}
\usepackage{ragged2e}
\newcolumntype{Y}{>{\centering\arraybackslash}X}

 \AtBeginDocument{
 \heavyrulewidth=.08em
 \lightrulewidth=.05em
 \cmidrulewidth=.03em
 \belowrulesep=.65ex
 \belowbottomsep=0pt
 \aboverulesep=.4ex
 \abovetopsep=0pt
 \cmidrulesep=\doublerulesep
 \cmidrulekern=.5em
 \defaultaddspace=.5em
}

%Paper formatting based on Higgs CP study by Andres Rios - commented sections from this paper

\begin{document}

\title{Study of Higgs to WW Coupling Measurement Performance through $e^+e^-\rig\nu\bar{\nu} b\bar{b}$ at Future
Circular Colider - Electron Positron}

\author{A.~Apyan}
\affiliation{Massachusetts Institute of Technology, Cambrigde, USA}
\author{M.~Klute}
\affiliation{Massachusetts Institute of Technology, Cambrigde, USA}
\affiliation{European Organization for Nuclear Research (CERN), Meyrin, CH}
\author{A.~Andriatis}
\affiliation{Massachusetts Institute of Technology, Cambrigde, USA}

\begin{abstract}
This investigation seeks to evaluate the potential performance capabilities of the measurment of Higgs to WW coupling and the total Higgs decay width at the FCC-ee - a future $e^+e^-$ collider. The signal process used in the investigation is $e^+e^-\rig\nu\bar{\nu} b\bar{b}$ through WW Fusion, and considers various backgrounds. Unlike previous studies, this investigation closely evaluates the effect of detector performance on the coupling uncertainty. The uncertainty on the measurement of  $\sigma_{\nu\bar{\nu}H}$ x BR $(H \rig b\bar{b})$ is found to be 2.3\% at $\sqrt{s} = 350 \gev$ and 3.8\% at $\sqrt{s} = 240 \gev$, leading to a model-independent uncertainty on Higgs to WW coupling of 0.8\% and 1.1\% respectively, and an uncertainty on the total Higgs decay width of 1.1\% and 1.4\%, respectively.
\end{abstract}

\maketitle
\tableofcontents
\newpage

%%%%%%%%%%%%%%%%%%%%%%%%%%%%%%%%%%%%%%%%%%%%%%
\section{Introduction}
\label{sec:intro}

% Why e+ e- collider
% Why measure accurately
% Why Higgs WW
% Why nu nu b b
% What previous studies have been done
% Conclusions

To continue the search for physics beyond the standard model, precision knowledge of standard-model particles and their properties is required to measure deviations at higher energy scales. In investigating the Higgs boson, evidence for new physics at an energy scale of 1 TeV is expressed in deviations of the Higgs boson coupling to gauge bosons and fermions of up to 5\% relative to Standard Model predictions, and scales as $1/{\Lambda^2}$ \cite{tlep}. It is therefore necessary to measure Higgs boson couplings to per-cent accuracy or better to be sensitive to new physics at 1 TeV, and to per-mil accuracy or better to be sensitive to new  physics at multi-TeV.

Among the possible machines proposed for a precision study of the standard model particles, the FCC-ee (Future Circular Collider electron-positron) stands out as the tool of choice. A circular $e^+e^-$ collider provides a high-luminosity higgs factory with a clean detection environment useful for studying various higgs properties as well as Z and W bosons and the top-quark \cite{TLEP}.

Constraining the Higgs to W boson coupling ($g_\text{HWW}$) and the total Higgs boson width ($\Gamma_\text{tot}$) is a top priority. 

One path for this measurement is through studying the process $e^+e^-\rig\nu\bar{\nu} b\bar{b}$. This final state occurs through two processes, Higgs-strahlung (HZ) (Figure \ref{fig:hz_feyn}) where the Z decays into a neutrino pair, and WW fusion (Figure \ref{fig:ww_feyn}).\\
\begin{figure}[!htb]
	\centering
	\begin{minipage}{.5\textwidth}
		\centering
				\begin{fmffile}{zh}
					\begin{fmfgraph*}(100,40)
						\fmfleft{i1,i2}
						\fmfright{o1,o2}
						\fmflabel{$e^+$}{i1}
						\fmflabel{$e^-$}{i2}
						\fmflabel{$H$}{o2}
						\fmflabel{$Z$}{o1}
						\fmf{fermion}{i2,v1,i1}
						\fmf{dashes}{v2,o2}
						\fmf{boson}{v2,o1}
						\fmf{boson,label=$Z^*$}{v1,v2}
					\end{fmfgraph*}
				\end{fmffile}
		\caption{Higgs-strahlung}
		\label{fig:hz_feyn}
	\end{minipage}%
	\begin{minipage}{0.5\textwidth}
		\centering
				\begin{fmffile}{ww}
					\begin{fmfgraph*}(200,40)
						\fmfleft{i1,i2}
						\fmfright{o1,o2,o3}
						\fmflabel{$e^+$}{i1}
						\fmflabel{$e^-$}{i2}
						\fmflabel{$\nu_{l}$}{o3}
						\fmflabel{$H$}{o2}
						\fmflabel{$\bar{\nu_{l}}$}{o1}
						\fmf{fermion}{i2,v1,o3}
						\fmf{fermion}{o1,v2,i1}
						\fmf{boson}{v1,v3,v2}
						\fmf{dashes}{v3,o2}
						\fmf{boson,label=$W^+$}{v1,v3}
						\fmf{boson,label=$W^-$}{v2,v3}
					\end{fmfgraph*}
				\end{fmffile}
		\caption{WW Fusion}
		\label{fig:ww_feyn}
	\end{minipage}
\end{figure}

The measurement of $\sigma_{\nu\bar{\nu}H}$ x BR $(H \rig b\bar{b})$ contributes to the calculation of the total Higgs boson width, given by
\begin{equation}
	\Gamma_\text{tot} = \frac{\Gamma(\text{H}\rig\text{W W})}{\text{BR}(\text{H}\rig\text{W W})}
\end{equation}

The uncertainty of the measurement of $\sigma_{\nu\bar{\nu}H}$ x BR $(H \rig b\bar{b})$, determined by fittting the WW Fusion profile to a Monte Carlo (MC) simulation of the $\nu\bar{\nu}b\bar{b}$ signal and backgrounds, can therefore be combined with previously published uncertainty calculations on Higgs branching ratios to determine the potential precision of the Higgs boson total width measurement at the FCC-ee.

Two center-of-mass energies are considered - $\sqrt{s} = 350 \gev$ and $\sqrt{s} = 240 \gev$. 240 GeV is the center-of-mass energy proposed for the FCC as a Higgs factory, which balances the Higgs production cross section with the FCC-ee luminosity profile as a function of energy. While the Higgs cross-section peaks at 255 Gev, the FCC-ee luminosity decreases with increasing energy due to bremsstrahlung, leading to a maximum luminosity at $\sqrt{s} = 240 \gev$. It is proposed to run the FCC-ee at $\sqrt{s} = 240 \gev$ for five years, leading to a total integrated luminosity of 10 ab$^-1$ with four interaction points. The FCC-ee proposal also includes a five-year run at the $t\bar{t}$ threshold of $\sqrt{s} = 350 \gev$ with an integrated luminosity of 2.6 ab$^-1$. For this investigation, while large luminosity is preferable and decreases statistical uncertainty, the shape separation between the WW Fusion profile and its most similar background, ZH, increases as the available phase space of the missing mass increases. While the missing mass of the ZH background clusters around the Z boson mass of 91 GeV, the missing mass of the WW Fusion signal peaks at around $\sqrt{s}$ - m$_{H}$. A larger difference between the shape of the WW Fusion signal and its backgrounds leads to a lower uncertainty in the signal fit to the simulated data, thereby decreasing the uncertainty of the measurements. 

In the course of this study, the effects of detector performance on the precision of coupling measurements is investigated, and compared against detectors simulated in previous studies investigating Higgs decays at the ILC \cite{Higgs ILC}, at TLEP \cite{TLEP} and at LEP3 using the CMS detector \cite{lep3}.

%%%%%%%%%%%%%%%%%%%%%%%%%%%%%%%%%%%%%%%%%%%%%%
\section{Objects}
\label{sec:samples}
% What is used to detect the higgs signal

The signal process investigated is $e^+e^-\rig\nu\bar{\nu} b\bar{b}$ through WW fusion. The visible decay products are two b-quarks, which hadronize into two jets of particles, and are the objects detected and reconstructed by the detector. Neutrinos escape the detector and are searched for through the profile of the missing four-momentum calculated from the two reconstructed jets.


%%%%%%%%%%%%%%%%%%%%%%%%%%%%%%%%%%%%%%%%%%%%%%
\section{MC Event generation and detector simulation}
\label{sec:samples}
% Which tools do we use to generate and simulate

The generation of signal and background events was performed through Whizard \cite{whizard} which is a next-to-leading order tool and includes initial state radiation. Parton showering and hadronization was done using Pythia 8 \cite{pythia}. Detector simulation was done using Delphes \cite{delphes}. A card was made to simulate an optimal detector for Higgs precision studies at the FCCee. 

The signal event produced is $\nu\bar{\nu}b\bar{b}$ through WW Fusion. The background events considered are $\nu\bar{\nu}b\bar{b}$ through ZH, $\nu\bar{\nu}c\bar{c}$, $\nu\bar{\nu}q\bar{q} (q\neq b,c)$, $q\bar{q}l^+l^-$, $q\bar{q}l^-\nu$, $q\bar{q}q\bar{q}$, $q\bar{q}$, and $q\bar{q}\gamma$.  

%%%%%%%%%%%%%%%%%%%%%%%%%%%%%%%%%%%%%%%%%%%%%%
\section{Candidate selection}
\label{sec:selection}
% what is the event selection 
% explain how this was optimized

To select the WW Fusion $\nu\bar{nu}b\bar{b}$ signal, first a sample is created which looks for two jets reconstruccted using the anti-$k_T$ algorithn wth a distance parameter of 1.5. The jets are b-tagged with an efficiency specified by !!!. Only events containing two jets which are both b-tagged are allowed. Events with isolated leptons are omitted. 

Next, kinematic cuts are applied based on the distribution of various properties. The visible mass of the system is required to be between 110 and 140 GeV since the visible mass from the signal comes only from the Higgs. The visible PT of the system is required to be greater than 15 GeV. The visible PZ is required to be less than 90 GeV. The number of charged tracks in the two jets is required to be between 10 and 30. The acoplanarity angle greater than 0.1, lab angle between 1.5 and 3, and CosTheta of the leading jet less than 0.95 further eliminates backgrounds. The visible energy of the system and recoil mass of the di-jet system are not considered because the uncertainty in the cross section and branching ratio of the Higgs bb system is given by the shape fit of the recoil mass, since at 350 GeV the missing mass has a large separation between the ZH and WW Fusion processes. 

%%%%%%%%%%%%%%%%%%%%%%%%%%%%%%%%%%%%%%%%%%%%%%
\section{Statistical method}
\label{sec:method}
% what is the observable used
% what is the statistical method

The uncertainty in the quanitity of interest, $\sigma_{\nu\bar{\nu}H}$ x Br $(H \rig b\bar{b})$ is found using a Maximum Likelihood tool from HiggsAnalysis/CombinedLimit \cite{higgsanalysis}. This measurement is then combined with other measurements necessary to arrive at a total higgs width in a coupling tool \cite{couplingtool}.

The Maximum Likelihood tool allows the specification of systematic uncertainties. The unertainties used are a lognormal systematic uncertainty of 2.6\% on luminosity and an individual lognormal uncertainty of 1\% on each of the backgrounds and signal.

The analysis gives a total uncertainty of 2.29\% in $\sigma_{\nu\bar{\nu}H} x Br(H \rig b\bar{b})$. Compare this to an uncertainty of 10.5\% at 240 GeV and 0.66\% at 500 GeV given by the ILD paper \cite{ILD}.

The total higgs width was calculated to be !!! and the higgs WW coupling is calculated as !!!.

%%%%%%%%%%%%%%%%%%%%%%%%%%%%%%%%%%%%%%%%%%%%%%
\section{Results}
\label{sec:results}
% discuss the results
% (compare results with previous studies

The uncertainty on the measurement of  $\sigma_{\nu\bar{\nu}H}$ x Br $(H \rig b\bar{b})$ is found to be 2.3\% at $\sqrt{s} = 350 \gev$ and 3.8\% at $\sqrt{s} = 240 \gev$, (compare to 0.6\% and 2.2\% in the TLEP paper) leading to a model-independent uncertainty on Higgs to WW coupling of 0.8\% and 1.1\%! respectively, and an uncertainty on the total higgs width of 1.1\% and 1.4\%, respectively (compare to 1.2\% and 2.4\% in the TLEP paper).

The results found in this analysis suggests an improvement in the potential precision of Higgs to WW coupling measurements using an FCCee - specific detector, and improves the physics case for a circular electron positron precision higgs factory. 

%%%%%%%%%%%%%%%%%%%%%%%%%%%%%%%%%%%%%%%%%%%%%%
\section{Conclusion}
\label{sec:conclusion}


This will be the conclusion.

\clearpage

%%%%%%%%%%%%%%%%%%%%%%%%%%%%%%%%%%%%%%%%%%%%%%
\begin{thebibliography}{99}

 \bibitem{fccee}
   CERN, The FCC-ee design study, \url{http://tlep.web.cern.ch/}
  
 \bibitem{madgraph}
   J.~Alwall {\it et al.},
   {\it JHEP} {\bf 07} (2014) 079.
   %[arXiv:1405.0301v2]
  
 \bibitem{pythia}
   T.~Sjostrand, S.~Mrenna, and P.Z.~Skands,
   Comput. Phys. Commun. {\bf 178} (2008) 852–867.
   %[arXiv:0710.3820v1]
 
 \bibitem{delphes}
   J.~de~Favereau {\it et al.} [DELPHES 3 collaboration],
   {\it JHEP} {\bf 02} (2014) 057.
   %[arXiv:1307.6346v3]
  

\end{thebibliography}

\end{document} 
